%===================================================
\chapter{Modèle de propagation et nature des sources aéroacoustiques}
%===================================================


On décrit ici les sources de bruit d'un avion à turboréacteur double flux, les mécanismes de génération de bruits en écoulement et les méthodes expérimentales permettant de reproduire ces sources ainsi que de les mesurer. \\




Le bruit peut être généralement décomposé en 4 composantes : \\
-une partie tonale générée par les composantes tournantes de la machine\\
-une partie cyclostationnaire induite par les composantes tournantes de la machine\\
-le bruit machine aléatoire \\
-le bruit hydrodynamique.\\


Séparer ces composantes dans le champ total mesuré permet de mieux comprendre la contribution de chaque source ou de chaque élément du réacteur, par exemple. Cette séparation est décrite dans le chapitre~\ref{debruitage}.




\section{Exemples de sources aéroacoustiques sur un avion}
%=============================================================

\todo[inline]{Un bref récap des sources est fait en intro de la thèse de G. Reboul et Simon B.}
\cite{Smith1989} décrit un très grand nombre de sources aéroacoustisques sur un avion. Elles peuvent être classées en 2 catégories : le bruit de moteur et le bruit aérodynamique. Le bruit aérodynamique est principalement généré par le train d'atterrissage et par les ailes. La principale source de bruit des ailes est liée aux volets à l'avant (becs de bord d'attaque) et à l'arrière. Ces volets sont des hypersustentateurs qui augmentent la portance qui génèrent localement beaucoup de bruit. Mais paradoxalement, leur présence contribuent fortement à la réduction du bruit des avions par le fait qu'ils favorisent un décollage rapide et un atterrissage à vitesse réduite.\\
Comme le montre l'image de \cite{Smith1989} \ref{smith}, les bruits du moteur sont d'origines diverses. Les moteurs doubles flux ont permis de fortement diminuer le bruit de jet, ce qui rend le bruit aérodynamique égal voire prépondérant sur le bruit de moteur en configuration d'approche (atterrissage).

\begin{figure}[!h]
 \begin{center}
  \includegraphics[width=0.5\textwidth]{img/smith_noise.png}
  \caption{Comparaison des sources de bruits d'un moteur simple flux (à gauche) et d'un moteur double flux (à droite). Image extraite de \cite{Smith1989}.\label{smith}}
 \end{center}
\end{figure}


\paragraph{Bruits tonaux}~\\
	$\bullet$ Fréquence de passage des pales (BPF : blade pass frequency) et ses harmoniques : bruit tonal, connu : $\omega = h Z \Omega$, $h=1,2,...$,  où $Z$ est le nombre de pales du rotor et $\Omega$ est sa fréquence de rotation. Les harmoniques qui apparaissent sont alors donnés par : $m=hZ-sV$, avec $m$ le numéro du mode azimutal, $V$ le nombre de pale du stator et $s=...,-1,0,1,...$\\
	$\bullet$ Bruit d'épaisseur (Blade thickness noise) : monopole, tonal. C'est le bruit généré par le déplacement du fluide autour des pales (présent à haute vitesse de rotation seulement).\\
	$\bullet$ Uniform inlet flow : peut être réduit en augmentant le nombre de pales\\

 

\paragraph{Bruits large bande}~\\
	\tbullet Flux inconstant : fluctuation stochastique de la vitesse du flux entrant génère un bruit large bande\\
	\tbullet Couche limite turbulente (TBL : Turbulent Boundary layer) : couche turbulent générée aux bords de fuite. Ce bruit peut être modélisé comme un ensemble de dipôles répartis sur la surface de l'aube.\\
	\tbullet Décollement de couche limite (Vortex shedding) : décollement de la couche limite (laminaire ou turbulente), ce qui change l'écoulement autour des pales\\
	\tbullet tip noise : bruit généré dans l'espacement entre les pales et le carter. Ce bruit augmente si l'espacement augmente. A noter que la vitesse à l'extrémité des pales étant grande, ce bruit peut être important.\\
	\tbullet Bruit de soufflante : dans les turboréacteurs double-flux principalement. Ref : thèse G. reboul\\


Bruit d'interaction rotor-stator : dominant ?\\
\todo[inline]{Compléter en lisant la thèse de Simon}

Les moteurs double flux ont permis de diminuer l'importance du bruit de jet, mais ont rajouté le bruit de soufflante.


\subsection{Bruit de jet}
Review de Tam sur le bruit de jet : \cite{TamReviewJet}.

page 86 de Smith : Description du bruit de jet\\
-small-scale Eddies (HF)\\
-large-scale Eddies (BF)\\
-mixing region\\
-shock noise\\
Sur un moteur à low-bypass-ration, le centre du jet sort à 500 m/s de la tuyère, et constitue la principale source de bruit du turbo réacteur.\\
Depuis les turboréacteurs double-flux, le jet chaud est entouré du jet froid issu de la soufflante.\\


\subsubsection{Shock-associated noise}
Review : \cite{Bailly2014}
Lorsque la pression en sortie de tuyère est supérieure à la pression ambiante, le jet est sous-détendu (traduction approx.) et des cellules de chocs se forment dans le jet pour qu'il s'adapte à la pression ambiante.
L'expansion de chocs génère un screech (tone) et un bruit large bande. Ce dernier est généré par l'interaction des tourbillons générés aux lèvres de la tuyère avec les cellules de chocs, et le screech viendrait d'un mécanisme de rétroaction des chocs sur la tuyère.\\

La fréquence de la bosse du spectre large bande diminue avec l'angle (pris par rapport à l'axe du jet aval) suivant la relation : 
\begin{equation*}
	    f_{\text{BBSAN}} = \frac{u_c}{L_{\text{shock}}(1-M_c \cos \theta)}
\end{equation*}
d'après Harper-Bourne \& Fisher. 
$u_c$ est la vitesse de convection, $L_{\text{shock}}$ la longueur des cellules de choc (qui dépend du diamètre du jet et du nombre de Mach du jet détendu) et $M_c=u_c  / c_\infty$.\\

Le screech peut être réduit en apportant des modifications aux lèvres de la tuyère : encoches (notched nozzle), rétrécissement ou épaississement. Le screech n'apparait que si la tuyère est axisymétrique (ce qui n'est pas le cas pour les avions civils).




\section{Physique de la Propagation acoustique en écoulement}
%=================================================================

Pour ces méthodes, on considère que la façon dont le son se propage est connue (matrice de transfert acoustique). Prendre en compte l'écoulement, sinon les sources apparaissent décalées vers l'aval (Amiet, par ex). 
Calibration de la matrice interspectrale avec et sans écoulement : ne nécessite pas de connaître la nature de l'écoulement. (S.Kroeber,K.Ehrenfried,L.Koop et A.Lauterbach, « In-flow calibration approach for improving beamforming accuracy) Mais contrainte expérimentale car coûteux en temps et surveillance des fluctuation de Temperature...\\


Pour comprendre le problème, il est nécessaire de rappeler les expression analytique de l'intensité acoustique rayonnée par un écoulement.

\subsection{Équations classiques de la mécanique des fluides}
\vspace{0.3cm}\paragraph{Conservation de la quantité de mouvement : Navier-Stokes}
\begin{equation}
	\frac{\dd \rho u_i}{\dd t} + \frac{\dd \rho u_i u_j}{\dd x_j} = -\frac{\dd p}{\dd x_i} + \rho g_i + \frac{\dd \tau_{ij}}{\dd x_j}
\end{equation}
\paragraph{Conservation de la masse}
\begin{equation}
 \frac{\dd \rho}{\dd t} + \frac{\dd \rho u_{i}}{\dd x_i} = 0
\end{equation}

\todo[inline]{Poursuivre après cours d'aéroacoustique}



%Nature des sources : différent niveaux, cohérent/incohérent, étendues/ponctuelles ? Hypothèses sur les sources : ondes planes/sphériques, par ex.

%Fonctions de Green en écoulement

%Spécificité de l'imagerie en écoulement turbulent

%Correction de l'écoulement : modèle d'Amiet ou modèle de Koop
%Étude de la correction à appliquer pour l'écoulement en beamforming : thèse Haddad.


%Source dipolaire/monopolaire : influence sur la méthode ?

%Nature du bruit de fond ?

%\todo[inline]{Prise en compte des réflexions sur le jet, interaction non-linéaire des sources, ...}

%===================================================
\section{Aspects expérimentaux}
%===================================================
\todo[inline]{Etat de l'art beamforming dans le doc \url{CEAS-special-issue-arrays_3-April-2017_error.pdf}}

Pour obtenir une représentation spatiale d'un champ stationnaire, les mesures peuvent être réalisées de plusieurs manières. Le ou les capteurs peuvent être déplacés dans l'espace pas à pas ou continûment (\cite{Comesana2013} pour un exemple de scan manuel, la position du capteur étant enregistrée par une vidéo) . Les mesures peuvent aussi être réalisées simultanément, moyennent l'utilisation d'une antenne fixe et d'une éventuelle carte d'acquisition multivoies. Pour caractériser un champ instationnaire à un instant donné, seules les mesures simultanées peuvent être réalisées.

\subsection{Mesures en soufflerie}
Les mesures en soufflerie sont plus simples à réaliser que les mesures en vol (échelle réduite, instrumentation facilitée, source fixe, ...) et permettent de réaliser les tests dans un environnement contrôlé. Cependant, pour assurer leur validité, il faut s'assurer que les conditions en soufflerie sont équivalentes aux conditions réelles. Les mesures peuvent être réaliser en soufflerie à veine close ou ouverte.
\paragraph{Veine close}
 Les veines closes permettent un meilleur contrôle des condition d'écoulement mais induisent un bruit de turbulence plus élevé que les veines ouvertes. Elles induisent également des réflexion sur les parois qu'il faut prendre en compte par un modèle de sources images, par exemple (The Reflection Canceller, Giudati ?).\\
 BiClean permet aussi de faire de la déréverbération. CLEAN-SC devrait aussi être capable de le faire...
 
 \paragraph{Veine ouverte}
 Les veines closes limitent les problèmes de réflexions et réduisent le bruit de fond, mais elles posent d'autres problèmes.\\
 Si l'antenne est placées en dehors de l'écoulement, les ondes provenants de sources placées dans l'écoulement sont déviées par la couche de cisaillement situées entre les sources et l'antenne. Cette déviation doit être prise en compte par une approche géométrique (Design and use of microphone directional arrays for aeroacoustic measurements. Humphreys ou bien Shear layer correction validation using a non-intrusive acoustic point source, Bahr) ou par une fonction de Green adaptée. Cette déviation peut également être corrigée à l'aide d'une mesure de calibration (Investigation  of  the  systematic  phase  mismatch in  microphone-array  Analysis, Koop).


\subsection{Mesures en vol}
Si les capteurs sont positionnés au sol, la principale problématique est de mettre en oeuvre une dédopplerisation. Ces mesures sont plutôt pratiquées pour connaître le bruit à l’atterrissage ou au décollage.\\
Pour des mesures du bruit aérodynamique, les capteurs peuvent être posés sur l'avion en régime iddle. Le contenu des mesures dépend alors grandement de la position des capteurs (proximité des turbomachines, des ailes, ...).



\subsection{Antenne}
Les mesures en présence d'un fort écoulement sont fortement marquée par le bruit de turbulence. Pour réduire ce bruit, différentes stratégies peuvent être mises en place. Les microphones peuvent être montés dans des cavités de manière à filtrer les petites longueurs d'ondes associées au bruit de turbulence. En veine ouverte, les microphones peuvent être placés en dehors du jet.  


différence entre antenne linéaire et antenne plane et antenne 3D.\\

acquisition(antenne, micro, accéléro, MEMS)/excitation (nature des sources)\\

ref sur l'influence de la position des micros : thèse antoine peillot\\






