\documentclass[12pt]{article}
\setlength{\columnsep}{2cm}

%\usepackage{cite} 
%\usepackage[round,authoryear,numbers]{natbib}
\usepackage[french]{babel}
\usepackage[utf8]{inputenc}
\usepackage{graphicx} %pour mettre des figures dans multicol avec l'environnement figure*

\usepackage[T1]{fontenc} % Use 8-bit encoding that has 256 glyphs
%\linespread{1.05} % Line spacing - Palatino needs more space between lines
%\usepackage{microtype} % Slightly tweak font spacing for aesthetics

\usepackage[hmarginratio=1:1,top=2cm, right=2cm]{geometry} % Document margins
\usepackage[textfont=it]{caption} % Custom captions under/above floats in tables or figures
\usepackage{booktabs} % Horizontal rules in tables
%\usepackage{float} % Required for tables and figures in the multi-column environment - they need to be placed in specific locations with the [H] (e.g. \begin{table}[H])
\usepackage{hyperref} % For hyperlinks in the PDF

\usepackage{bm}
\usepackage{amsfonts}
\usepackage{amsmath}
\usepackage{amssymb}
\usepackage{tabularx}

%\usepackage{titlesec} % Allows customization of titles
%\renewcommand\thesection{\Roman{section}} % Roman numerals for the sections
%\renewcommand\thesubsection{\arabic{section}.\arabic{subsection}} % Roman numerals for subsections
%\titleformat{\section}[block]{\bfseries\centering}{\thesection.}{1em}{}[{\titlerule[1.2pt]}] % Change the look of the section titles
%\titleformat{\subsection}[block]{\bfseries}{\thesubsection.}{1em}{} % Change the look of the section titles
%\renewcommand\thesubsubsection{\small{\arabic{section}.\arabic{subsection}.\arabic{subsubsection}}}
%\titleformat{\subsubsection}[block]{\bfseries}{\thesubsubsection}{0.5em}{}
%\titleformat*{\paragraph}{\vspace{-0.3cm}\small\bfseries}

\newcommand{\tbullet}{$\vcenter{\hbox{\tiny$\bullet$}}~$}

\usepackage{fancyhdr} % Headers and footers
\pagestyle{fancy} % All pages have headers and footers
\fancyhead{} % Blank out the default header
%\fancyfoot{} % Blank out the default footer
\renewcommand{\headrulewidth}{0pt} %pour enlever la ligne du header
%\fancyhead[C]{titre, date, noms...	} % Custom header text
%\fancyfoot[RO,RE]{\thepage} % Custom footer text
%\fancyfoot[LO,LE]{A. DINSENMEYER, \today}
%\renewcommand{\footrulewidth}{0.4pt} 

%modif des espacement avant et après l'environnement equation
\let\oldequation=\equation
\let\endoldequation=\endequation
\renewenvironment{equation}{\vspace{-0.2cm}\begin{oldequation}}{\vspace{-0.2cm}\end{oldequation}}
 
%agrandissement de la zone de texte
%\addtolength{\oddsidemargin}{-1cm}
%\addtolength{\evensidemargin}{-1cm}
%\addtolength{\textwidth}{2cm}
%\addtolength{\topmargin}{-0.7cm}
\addtolength{\textheight}{1cm}

\usepackage{color}
\usepackage[color=blue!20]{todonotes}
\usepackage{mathtools}

%pour écrire du pseudo code :
\usepackage{algorithm}
\usepackage{algorithmic}

\usepackage{hyperref}
\hypersetup{
     colorlinks   = true,
     citecolor    = blue!90
}

\newcommand{\dd}{\partial}
\newcommand{\ok}{ \textcolor{orange}{\bfseries \textsc ok }}


\usepackage{subcaption}
\usepackage{tabulary}
\usepackage{pgfplots} 

%pour la biblio en fin de page
\usepackage{filecontents}


%----------------------------------------------------------------------------------------
%	TITLE SECTION
%----------------------------------------------------------------------------------------

\title{ {\fontsize{14pt}{14pt}\selectfont Rapport de séminaire, ED MEGA par Alice Dinsenmeyer} \\[1cm]
\fontsize{18pt}{18pt}\selectfont\textbf{Validation numérique et expérimentale du beamforming fonctionnel pour la quantification des sources}} % Article title
\author{
\large{Présenté par Valentin Baron, doctorant MicrodB}\\% Your name %\thanks{}
%\normalsize École doctorale MEGA \\ % Your institution
%\normalsize \href{mailto:john@smith.com}{john@smith.com} % Your email address
\vspace{-5mm}
}
\date{le 14/12/17}

%----------------------------------------------------------------------------------------

\begin{document}
\maketitle
 En 2014, Dougherty propose une méthode d'imagerie acoustique appelée le beamforming fonctionnel (BF)\cite{doug}. Cette méthode fait partie des méthodes de formation de voies pour lesquelles le vecteur de pointage est construit à partir d'un prétraitment des données. On peut citer dans cette famille la méthode beamforming orthogonal ou encore la méthode Capon. Le BF peut être vu comme la généralisation de cette dernière.
 
\section{Principe}
Cette méthode a pour but d'améliorer la dynamique et la résolution des images, comparé au beamforming conventionnel. Le principe de cette méthode est décrit ici.\\
La matrice interspecrale microphonique est décomposée en valeurs propres :
\begin{equation}
    \bm{S_{pp}}=\bm{USU'}.
\end{equation}
Ses valeurs propres sont ensuites élevées à la puissance $1/\nu$ (d'où le nom de la méthode). Le bemaforming classique est ensuite appliqué, avec une mise à la puissance $\nu$ pour rétablir l'intensité : 
\begin{equation}
    \bm{S_{ii}} = \frac{(\bm{g'}\bm{S_{pp}}^{\frac{1}{\nu}}\bm{g})^\nu}{\|\bm{g}\|^2}
\end{equation}
Là où l'intensité est maximale, le dénominateur vaut $1$. La mise à la puissance a donc pour effet d'amplifier les grande valeurs propres et de diminuer les petites valeurs propres (associées au bruit, par exemple).

\section{Résultats}
Cette méthode dépent donc du paramètre $\nu$. Valentin a présenté les résultats suivants : 
\begin{itemize}
    \item plus le RSB est grand, plus la dynamique peut être améliorée avec $\nu$,
    \item l'augmentation de $\nu$ entrtaîne une sous-estimation du niveau de la source,
    \item  l'augmentation de la fréquence d'étude entrtaîne une sous-estimation du niveau de la source.
\end{itemize}
Il y a donc un compromis à trouver entre la diminution des lobes secondaires et la dégradation de la quantification.

 
\section{Analyse}
Dans l'ensemble, la localisation est améliorée par cette méthode. Cependant, les performances de quantification restent incertaines. En pratique, le maximum d'intensité ne donne pas forcément $1$ et les pics sont alors affessé par la mise à la puissance. Cette méthode demande de bien connaître le modèle de propagation, ce qui n'est pas toujours le cas en pratique.\\

L'algorithme est simple à implémenter et peu coûteux (le principla coût vient de la SVD).//

\section{Question/Remarques}
Cette méthode aurait déjà été proposé dans les années 1980 par Laguna ?\\
Méthode de type Analyse en Composatnes Principales. 
 
\medskip
\bibliographystyle{plainnat}

\begin{thebibliography}{9}
\bibitem{doug} 
 {Dougherty, R. P.},
\textit {Functional beamforming for aeroacoustic source distributions},
 {AIAA paper},
 {2014},
 {vol. 3066}.
\end{thebibliography}
\end{document}
