\documentclass[12pt,a4paper,answers]{article}
\usepackage[lmargin=1.5cm,tmargin=1.2cm,bmargin=1.5cm,rmargin=1.5cm]{geometry}
\usepackage[french]{babel}
%\usepackage{fontspec}
\usepackage{graphicx}
\usepackage{amsthm,amssymb,amsfonts,mathtools}
\usepackage[utf8]{inputenc}
\usepackage{multicol}
\usepackage{multirow}
\usepackage{amssymb}
\usepackage{array}
\usepackage{graphicx}
%\setlength{\columnseprule}{0.5pt}
\usepackage{url}


\newcounter{exo}[section]
\setcounter{exo}{0}
\newenvironment{exo}[1]{\stepcounter{exo}\\[1em] \fbox{\textbf{Exercice \theexo : #1}}\\[0.5em]}{}

%Pour le mode exam
\usepackage[auto-label]{exsheets}
\SetupExSheets{
	solution/print=true,
	headings=block-subtitle,
	counter-within = exo,
 	counter-format = \theexo.qu
}

\newcommand{\sol}[1]{\PrintSolutionsTF{~\\ \colorbox{gray!10}{\parbox{01\linewidth}{#1}}}}
\renewcommand{\d}{\mathrm{d}}
\newcommand{\e}{\mathrm{e}}

\everymath{\displaystyle}



\usepackage{icomma} %pour le bon espacement dans le nombres décimaux


%%%%%%%%%%%%%%%%%%%%%%%%%%%%%%%%%%%%%%%%%%
\begin{document}
%%%%%%%%%%head%%%%%%%%%%%%%%%%%%%%%%%
%\twocolumn[ 
%  \begin{@twocolumnfalse}
% \begin{center}
% \fbox{\Large \textsc{ Exercices : Calcul matriciel}}
%\end{center}
%    \hrulefill
%\end{@twocolumnfalse}
 % ]
\begin{center}
	Alice Dinsenmeyer ~~~\url{alice.dinsenmeyer@insa-lyon.fr} \\[1ex]
	\fbox{\Large \textsc{ Exercices supplémentaires : Calcul intégral}}
\end{center}

 %%%%%%%%%%%%%%%%%%%%%%%%%%%%%%%%%%%%%%%%%%%%%%%%%%%%
%\begin{multicols}{2}
%\end{multicols}

%%%%%%%%%%%%%%%%%
%
% Primitives
%
%%%%%%%%%%%%%%%%%
\indent\exo{Primitives}
Donner les primitives des fonctions suivantes :
%
\begin{tasks}(3)
        	\task  $f(x)=3x^2+2x+1$
        	\sol{$ F(x)=x^3+x^2+x+ c$}
        	\task  $f(x)=\sin(x)$
        	\sol{$ F(x)=-\cos(x)+ c$}	
	\task  $f(x)=\frac{1}{x}$
        	\sol{$ F(x)=\ln(|x|)+ c$}
	\task  $f(x)=x-\frac{1}{x^2}$
        	\sol{$ F(x)=\frac{x^2}{2}+ \frac{1}{x}+ c$}
        	\task  $f(x)=-x^2+x$
        	\sol{$ F(x)=-\frac{x^3}{3}+\frac{x^2}{2}  + c$}        	
        	\task  $f(x)=\frac{1}{x^3}$
        	\sol{$ F(x)=-\frac{1}{2x^2}+ c$}
        	\task  $f(x)=\frac{x^4+1}{x^2}$
        	\sol{$ F(x)=\frac{x^4-3}{3x}+ c$}
        	\task  $f(x)=3\sin(x)+2\cos(x)$
        	\sol{$ F(x)=-3\cos x+2\sin x+ c$}
        	\task  $f(x)=2(2x+1)^3$
        	\sol{$ F(x)=\frac{(2x+1)^4}{4}+ c$}
        	\task  $f(x)=(3x+1)^{-5}$
        	\sol{$ F(x)=-\frac{(3x+1)^{-4}}{12}+ c$}
        	\task  $f(x)=(-2x+1)^5$
        	\sol{$ F(x)=-\frac{(-2x+1)^4}{8}+ c$}
        	\task  $f(x)=\frac{2x+1}{(x^2+x+1)^4}$
        	\sol{$ F(x)=-\frac{1}{3(x^2+x+1)^3}+ c$}
        	        	\task  $f(x)=\sin(x)\cos^3(x)$
        	\sol{$ F(x)=-\frac{\cos^4(x)}{4}+ c$}
        	        	\task  $f(x)=\frac{\ln^2(x)}{x}$
        	\sol{$ F(x)=\frac{\ln^3(x)}{3}+ c$}
        	        	\task  $f(x)=\frac{1}{\sqrt{x+1}}$
        	\sol{$ F(x)=2\sqrt{x+1}+ c$}
        	        	\task  $f(x)=\frac{3x}{\sqrt{x^2+1}}$
        	\sol{$ F(x)=3\sqrt{x^2+1}+ c$}
        	        	\task  $f(x)=\frac{1}{x^2\sqrt{1+\frac{1}{x}}}$
        	\sol{$ F(x)=-2\sqrt{1+\frac{1}{x}}+ c$}
        	        	\task  $f(x)=3\sin(3x+\frac{\pi}{2})$
       	\sol{$ F(x)=\sin(3x)+ c$}
        	        	\task  $f(x)=x\cos(x^2+\pi)$
        	\sol{$ F(x)=-\frac{1}{2}\sin(x^2)+ c$}
        	        	\task  $f(x)=\frac{\sin(\sqrt{x})}{\sqrt{x}}$
       	\sol{$ F(x)=-2\cos(\sqrt{x})+ c$}
        	        	\task  $f(x)=\frac{2x^2+3x+5}{x}$
        	\sol{$ F(x)=x^2+3x+5\ln(x)+ c$}
        	        	\task  $f(x)=\frac{\ln(x)}{x}$
        	\sol{$ F(x)=\frac{\ln^2(x)}{2}+ c$}
        	        	\task  $f(x)=\frac{\e^x}{\e^x+1}$
        	\sol{$ F(x)=\ln(\e^x+1)+ c$}
        	        	\task  $f(x)=\frac{1}{\e^{2x}}$
        	\sol{$ F(x)=-\frac{1}{2\e^{2x}}+ c$}
        	        	\task  $f(x)=\frac{\sin(x)}{2+\cos(x)}$
        	\sol{$ F(x)=-\ln(2+\cos(x))+ c$}
        	        	\task  $f(x)=\frac{x^3}{1+x^2}$
        	\sol{$ F(x)=\frac{x^2}{2}-\frac{1}{2}\ln(1+x^2)+ c$}
\end{tasks}

%%%%%%%%%%%%%%%%%
%
% Intégrales
%
%%%%%%%%%%%%%%%%%
\indent \exo{Calcul d'intégrales}
Calculer les intégrales suivantes :
\begin{tasks}(2)
        	\task  $I=\int_0^3 (x+4)\d x$
        	\sol{$I=\left[ \frac{x^2}{2} + 4x\right]_0^3 = \frac{9}{2}+12$}
        	\task  $I=\int_{-1}^{1}(2t^2-1)\d t$
        	\sol{$I=\left[\frac{2t^3}{3}-t \right]_{-1}^1=\frac{4}{3}-2$}
        	\task  $I=\int_1^2 \frac{3}{\sqrt{t}}\d t$
        	\sol{$I=\left[6\sqrt{t} \right]_1^2=6(\sqrt{2}-1)$}
        	\task  $I=\int_0^\pi \sin(t)\d t$
        	\sol{$I=\left[-\cos(t)\right]_0^\pi=2$}
        	\task  $I=\int_0^1(2x+3)(x^2+3x-5)\d x$
        	\sol{$I=\frac{1}{2}\left[(x^2+3x-5)^2 \right]_0^1=-12$}
        	\task  $I=\int_{-1}^1\frac{2t+1}{(t^2+t+1)^2}\d t$
        	\sol{$I=\left[\frac{-1}{t^2+t+1} \right]_{-1}^1=\frac{2}{3}$}
        	\task  $I=\int_0^{\pi}\sin^2(t)\d t$
        	\sol{$I=\frac{1}{2}\left[x-\frac{\sin(2x)}{2} \right]_0^\pi=\frac{\pi}{2}$}
        	\task  $I=\int_{-1}^0\frac{2t}{t^2+1}\d t$
        	\sol{$I=\left[\frac{-1}{t^2+1} \right]_{-1}^0=-\frac{1}{2}$}
        	\task  $I=\int_1^2 x\ln x \d x$
        	\sol{Poser $u'=x$ et $v=\ln x$. $I=2\ln2-\frac{3}{4}$}
        	\task  $I=\int_0^1(2x+1)\e^x \d x$
        	\sol{Poser $u=2x+1$ et $v'=\e^x$. $I=1+\e$}
        	\task  $I=\int_1^{\e}\frac{\ln x}{x^2} \d x$
        	\sol{Poser $u=\ln x$ et $v'=\frac{1}{x^2}$. $I=\frac{\e-2}{\e}$}  
        	\task  $I=\int_1^x \ln t \d t$
        	\sol{$I=\left[t \ln t -t\right]_1^x=x\ln x-x+1$}          
\end{tasks}

\indent\exo{Primitives}
Donner les primitives des fonctions suivantes :
%

\begin{tasks}(2)
        	\task  $f(x)=\frac{1}{x^4-x}$
        	\sol{$ F(x)=\int \frac{x^2}{x^3-1}-\frac{1}{x}\d x\\ = \frac{1}{3}\ln(x^3-1)-\ln(x)+c$}
        	\task  $f(x)=\frac{3x+1}{x^2-1}$
        	\sol{$ F(x)=\int \frac{1}{1+x}+\frac{2}{x-1} \d x\\ = 2\ln(x-1) + \ln(x+1)+c$}
\end{tasks}

\indent\exo{Changement de variables}
Calculer les intégrales suivantes :
%

\begin{tasks}(1)
	\task  $I=\int_0^1\sqrt{1-x^2}\d x $
	\sol{Poser le changement $x=\sin(u)$. Alors $I=\int_0^{\frac{\pi}{2}}\sqrt{1-\sin^2u}\cos u~ \d u =\left[ \frac{\sin(2u)}{2} + \frac{u}{2}\right]_0^{\frac{\pi}{2}}= \frac{\pi}{4}$}
	\task  $I=\int_0^{\frac{\pi}{4}}\frac{\tan x}{\cos x \left( \cos x + \sin x \right)} \d x $
	\sol{Poser le changement $u=\tan x$. Alors $I=\int_0^1\frac{u}{u+1} \d u = 1- \ln 2$}
\end{tasks}



%%%%%%%%%%%%%%%%%%%%%%%%%%%%%%%%%%%%%%%%%%%%%%%%%%%%%%%%%%%%%%%%%%%%%
%\begin{center}
%	\fbox{\Large \textsc{ Exercices supplémentaires}}
%\end{center}


%%%%%%%%%%%%%%%%%%%%%%%%%%%%%%%%%%%%%%%%%%%%%%%%%%%%%%%%%%%%%%
\end{document}