\documentclass[12pt,a4paper,answers]{article}
\usepackage[lmargin=1.5cm,tmargin=1.2cm,bmargin=1.5cm,rmargin=1.5cm]{geometry}
\usepackage[french]{babel}
%\usepackage{fontspec}
\usepackage{graphicx}
\usepackage{amsthm,amssymb,amsfonts,mathtools}
\usepackage[utf8]{inputenc}
\usepackage{multicol}
\usepackage{multirow}
\usepackage{amssymb}
\usepackage{array}
\usepackage{graphicx}
%\setlength{\columnseprule}{0.5pt}
\usepackage{url}


\newcounter{exo}[section]
\setcounter{exo}{0}
\newenvironment{exo}[1]{\stepcounter{exo}\\[1em] \fbox{\textbf{Exercice \theexo : #1}}\\[0.5em]}{}

%Pour le mode exam
\usepackage[auto-label]{exsheets}
\SetupExSheets{
	solution/print=true,
	headings=block-subtitle,
	counter-within = exo,
 	counter-format = \theexo.qu
}

\newcommand{\sol}[1]{\PrintSolutionsTF{~\\ \colorbox{gray!10}{\parbox{01\linewidth}{#1}}}}
\newcommand{\ddroit}{\mathrm{d}}




\usepackage{icomma} %pour le bon espacement dans le nombres décimaux


%%%%%%%%%%%%%%%%%%%%%%%%%%%%%%%%%%%%%%%%%%
\begin{document}
%%%%%%%%%%head%%%%%%%%%%%%%%%%%%%%%%%
%\twocolumn[ 
%  \begin{@twocolumnfalse}
% \begin{center}
% \fbox{\Large \textsc{ Exercices : Calcul matriciel}}
%\end{center}
%    \hrulefill
%\end{@twocolumnfalse}
 % ]
\begin{center}
	Alice Dinsenmeyer ~~~\url{alice.dinsenmeyer@insa-lyon.fr} \\[1ex]
	\fbox{\Large \textsc{ Exercices : Calcul intégral}}
\end{center}
D'autres exercices corrigés se trouvent à l'adresse : \url{http://gecif.net/articles/mathematiques/integration/}
 %%%%%%%%%%%%%%%%%%%%%%%%%%%%%%%%%%%%%%%%%%%%%%%%%%%%
%\begin{multicols}{2}
%\end{multicols}

%%%%%%%%%%%%%%%%%
%
% Primitives
%
%%%%%%%%%%%%%%%%%
\indent\exo{Primitives usuelles}
Donner les primitives des fonctions suivantes et vérifier le résultat par dérivation :
%a changer pour sujet corrigé	3
\begin{tasks}(2)
        	\task  $\displaystyle f(x)=\sqrt{x}$ \sol{$\displaystyle F(x)=\frac{2}{3}x\sqrt{x} + c$}
	\task $\displaystyle  f(x)=1-x^2$  \sol{$\displaystyle F(x)=x-\frac{x^3}{3} + c$}
	\task $\displaystyle f(x)=e^{-2x} + e^{4x}$  \sol{$\displaystyle F(x)=-\frac{e^{-2x}}{2} + \frac{e^{4x}}{4} + c$}
	\task  $\displaystyle f(x)=3 + \sin(x)-\cos(x)$  \sol{$\displaystyle F(x) = 3x -\cos(x)-\sin(x) + c$}
	\task $\displaystyle f(x)=\frac{1}{2x}-\frac{2}{x^2}+\frac{3}{\sqrt{x}}$  \sol{$\displaystyle F(x)=\frac{\ln|x|}{2}+\frac{2}{x}+6\sqrt{x}+c$}
	\task $\displaystyle f(x)=\frac{1-x^2}{1+x^2}$  \sol{Écrire $f(x)=-1 + \frac{2}{1+x^2}$. \\ On trouve alors $ F(x)=-x + 2\operatorname{atan}(x)+c$}
	\task $\displaystyle f(x)=(3x-1)^6$ \\ \textit{($f$ est de la forme $g(u)u'$)} \sol{ $\displaystyle F(x)=\frac{(3x-1)^7}{21}+c$}
	\task $\displaystyle f(x)=\cos(3x-1)$  \sol{$\displaystyle F(x)=\frac{\sin(3x-1)}{3}+c$}
	\task \ $\displaystyle f(x)=\frac{x^2}{\sqrt{x^3-1}}$  \\ \textit{($f$ est de la forme $u'/\sqrt{u}$)} \sol{$\displaystyle F(x)=\frac{2 \sqrt{x^3-1}}{3} +c$}
\end{tasks}

%%%%%%%%%%%%%%%%%
%
% Intégrales
%
%%%%%%%%%%%%%%%%%
\indent \exo{Calcul d'intégrales}
Calculer les intégrales suivantes :
%a changer pour sujet corrigé	3
\begin{tasks}(2)
	\task $\displaystyle I=\int_{0}^{\frac{\pi}{2}}  x + \sin(x) \ddroit x $ 
	\sol{$I=\frac{\pi^2}{8} + 1$}
	\task $\displaystyle I=\int_2^3 \ln 2 + x \ddroit x $ 
	\sol{$I=\ln 2 + \frac{5}{2}$}
	\task $\displaystyle I=\int_0^2 \sqrt{|1-x|} \ddroit x$ 
	\sol{$\displaystyle I=\int_0^1 \sqrt{1-x} \ddroit x + \int_1^2 \sqrt{x-1} \ddroit x = \frac{2}{3} + \frac{2}{3}$}
\end{tasks}



%%%%%%%%%%%%%%%%%
%
% Applications diverses
%
%%%%%%%%%%%%%%%%%

\indent \exo{Applications diverses}
\begin{question}[name=~,subtitle=\parbox[t]{0.9\linewidth}{\normalfont Trouver la fonction dont la tangente est donnée par $\displaystyle x^3 -\frac{2}{x^2} + 2$ en tout $x$ et dont la courbe passe par le point $(1,3)$.}]	
\vspace{1em}
\end{question}
\sol{La tangente est la dérivée de $f$, donc $$ f'(x)=x^3 -\frac{2}{x^2} + 2$$
	$f(x)$ est la primitive de $f'$ :
	$$f(x)=\int f'(x) \ddroit x  = \frac{x^4}{4}+\frac{2}{x}+2x + C$$
	Or, la courbe de cette fonction passe par $(1,3)$, donc $C$ est telle que $f(1)=3$. On a donc $C=-\frac{5}{4}$. Finalement, la fonction recherchée $f$ est $\displaystyle f(x)= \frac{x^4}{4}+\frac{2}{x}+2x -\frac{5}{4}$.
}
	
\begin{question}[name=~,
subtitle={\normalfont On considère la fonction $g$ définie sur $[0,8]$ par :}]
	$\displaystyle \left\{ \begin{array}{l}
	g(x) = 4 \text{ si } x \in [0,3]\\
	g(x) = 6 \text{ si } x \in~ ]3,5]\\
	g(x) = -2 \text{ si } x \in~ ]5,8]\\
	\end{array} \right.$.	
%a changer pour sujet corrigé	2
	\begin{tasks}(1)
		\task Calculer $\displaystyle I=\int_0^8 g(x) \ddroit x $.
		\sol{~\\ En utilisant la relation de Chasles, $I=18$.}
		\task Calculer la valeur moyenne de $g$ sur $[0,8]$.
		\sol{~\\ La moyenne est donnée par $\displaystyle \frac{1}{8-0}\int_0^8 g(x)\ddroit x  = \frac{9}{4}$.}
	\end{tasks}
\end{question}

\begin{question}[name=~,subtitle={\normalfont Accélération d'un objet}]
	L'accélération d'un objet est donnée par l'équation $a(t)=3t$. Elle part de la position \hbox{$x(t=0)=4$}~m et sa vitesse initiale est $v(t=0) = 2$~m/s. 
%a changer pour sujet corrigé	3
	\begin{tasks}(1)
		\task Calculer l'accélération à 5 s.\\
		\sol{$a(t=5)=15$ m/s$^2$}
		\task Calculer la vitesse à 5 s.\\
		\sol{$\displaystyle v(t)-v(t=0)=\int_0^t a(t) \ddroit t$ donc $\displaystyle v(t)=\frac{3t^2}{2}+v(t=0)$, $v(t=5)=39,5$ m/s}
		\task Calculer la position à 5 s.\\
		\sol{De même, $\displaystyle x(t)=\int_0^t v(t) \ddroit t + x(t=0)$. $x(t=5)=76,5$ m.}
	\end{tasks}	
\end{question}

\begin{question}[name=~,subtitle={\normalfont Soit la fonction $f$ définie sur $[0;1]$ par $f(x)=x(1-x^2)$.}]
	Soit $D$ le domaine constitué de l'ensemble des points $M(x,y)$ tels que \hbox{$0 \leq x \leq 1$ et $0\leq y \leq x(1-x^2)$}.
	\begin{tasks}
		\task Calculer l'aire du domaine $D$
		\sol{$\displaystyle A=\int_0^1 f(x) \ddroit x = \left[ \frac{x^2}{2}\right]_0^1 - \left[ \frac{x^4}{4}\right]_0^1 = \frac{1}{4}$}
		\task Existe-t-il une droite $(d)$ passant par l'origine et partageant le domaine $D$ en deux parties de même aire ?
		\sol{On cherche une droite d'équation $y=ax$ qui coupe la courbe de $f$ : 
		\begin{equation*}
			x(1-x^2)=ax~~~~\Leftrightarrow  ~~~~ x=\sqrt{1-a}			
		\end{equation*}
		Le point d'intersection est $M(\sqrt{1-a},a\sqrt{1-a})$.\\
		Tracer la courbe de $f$, vérifier quelle est positive.\\
		L'aire entre $(d)$ et la courbe de $f$ doit être égale à $1/8$ : 
		\begin{equation*}
        			\int_0^{\sqrt{1-a}} f(x)\ddroit x  - \int_{0}^{\sqrt{1-a}} ax \ddroit x  = \frac{1}{8}
		\end{equation*}
		Cela revient à résoudre $\displaystyle \frac{a^2}{4}-\frac{a}{2} + \frac{1}{8}=0$, ce qui a pour solution $a_1=1+\frac{1}{\sqrt{2}}$ et $a_2 = 1-\frac{1}{\sqrt{2}}$. On remarque que $\displaystyle M(\sqrt{1-a_1},a_1\sqrt{1-a_1})$ a des coordonnées complexes, donc cette valeur de $a$ ne permet pas de partager $D$ comme demandé. La droite $(d)$ recherchée est donc d'équation $y=a_2x$.
		}
	\end{tasks}	
\end{question}


%%%%%%%%%%%%%%%%%
%
% IPP
%
%%%%%%%%%%%%%%%%%

\indent \exo{Intégration par partie}
\begin{question}[name=~,subtitle={\normalfont Calculer les intégrales suivantes :}]
%a changer pour sujet corrigé	4
	\begin{tasks}(1)
		\task $\displaystyle I=\int_0^{\pi} x\sin x \ddroit x $
		\sol{Poser $u=x$, $u'=1$ et $v=-\cos x$, $v'=\sin x$. Faire une IPP. $I=\pi$.}
		\task $\displaystyle I=\int_{1}^{2}\ln x \ddroit x $
		\sol{Poser $u=x$, $u'=1$ et $v=\ln x$, $v'=1/x$. Faire une IPP. En déduire que les primitives de $\ln x$ sont $x\ln x -x +c$. Enfin, $I=\ln4 -1$}
		\task $\displaystyle I=\int_0^{\pi} e^x \cos(x)\ddroit x $
		\sol{
		Poser $u=e^x$,  $u'=e^x$ et $v=\sin(x)$, $v'=cos(x)$. Faire une IPP.\\
		Refaire une IPP sur le terme $\int_0^{\pi}v u' \ddroit x $, en posant $w=e^x$, $w'=e^x$ et $y=-\cos x$, $y'=\sin x$.\\
		En déduire que $-I=e^{\pi}+1 + I$ et donc que $I=\frac{-e^{\pi}-1}{2}$.	
		}
		\task $\displaystyle I=\int_0^1 (3-2x)e^{-x} \ddroit x$
		\sol{Poser $u'=e^{-x}$ et $v=3-2x$. Faire une IPP.  $I=1+e^{-1}$. }
	\end{tasks}
\end{question}

%%%%%%%%%%%%%%%%%
%
% Changement de variables
%
%%%%%%%%%%%%%%%%%
\indent \exo{Changement de variables}
\begin{question}[name=~,subtitle={\normalfont Calculer les intégrales suivantes :}]	
%a changer pour sujet corrigé	2
	\begin{tasks}(1)
		\task $\displaystyle I=\int_0^1 \frac{1}{(1+x)^2}\ddroit x $
		\sol{Changement de variable : $y=1+x$, soit $x=y-1$. $\displaystyle I=\int_1^2 \frac{1}{y^2}\ddroit y     = \frac{1}{2}$.}
		\task $\displaystyle I = \int_{0}^{\frac{1}{2}} \frac{x}{\sqrt{1-x^2}}\ddroit x $
		\sol{Poser le changement de variable $u=1-x^2~~~\Rightarrow~~~~\ddroit u=-2x\ddroit x $. En déduire que \hbox{$\displaystyle\int_0^{\frac{1}{2}} \frac{x}{\sqrt{1-x^2}}\ddroit x =\int_{0,75}^{1}\frac{1}{2\sqrt{u}}\ddroit u$} et donc que $I=1-\sqrt{0,75}\approx 0,13.$}
	\end{tasks}
\end{question}

\begin{question}[name=~,subtitle={\normalfont Montrer que $\displaystyle \int \tan x \ddroit x  = -\ln |\cos x|$.}]
\end{question}
\sol{Faire le changement de variable $u=\cos x$.}

\begin{question}[name=~, subtitle={\normalfont Trouver les primitives des fonctions suivantes :}]

	\begin{tasks}(2)
		\task $f(x)=x e^{x^2}$
		\sol{Changement de variable : $u=x^2$. \\$F(x)=\frac{e^u}{2}+c$}
		\task $f(x)=(x \ln x)^{-1}$
		\sol{Changement de variable : $u= \ln x$. \\$F(x)=\ln|u|+c$.}
	\end{tasks}
\end{question}


%%%%%%%%%%%%%%%%%
%
% Fractions
%
%%%%%%%%%%%%%%%%%
\indent \exo{Fractions rationnelles}
Une fraction rationnelle est le quotient d'un polynôme P par un polynôme Q. Pour pouvoir en calculer une primitive, il faut la décomposer en une somme de fonctions dont on sait calculer les primitives. 

\begin{question}[name=~,subtitle={\normalfont On souhaite trouver une primitive de $ \displaystyle f(x)=\frac{2x+3}{(x-2)(x+5)}$}]
\vspace{1em}
	\begin{tasks}
		\task Déterminer les réels $a$ et $b$ tels que $\displaystyle f(x)=\frac{a}{x-2}+\frac{b}{x+5}$.
		\sol{$a=b=1$}
		\task En déduire une primitive de $f$.
		\sol{$F(x)=\ln(|x-2|)+\ln (|x+5|)+c$}
	\end{tasks}
\end{question}

\begin{question}[name=~,subtitle={\normalfont En utilisant la méthode de la question précédente, déterminer les primitives de }]
$$\displaystyle f(x)=\frac{2x-2}{x^2 -3x +2}$$
\end{question}
\sol{Remarquer que $x^2-3x+2 = (x-1)(x-2)$.
Écrire $\displaystyle f(x)=\frac{a}{x-1}+\frac{b}{x-2}$ et calculer que $a=0$ et $b=2$.
En déduire que $F(x)=2 \ln(|x-2|) + c$.}
	
\vspace{1em}
\begin{question}[name=~, subtitle={\normalfont Soit $\displaystyle f(t)=\frac{t^2 -1}{(t+3)^4}$. On souhaite calculer $\displaystyle I=\int_{-1}^{1} f(t)  \ddroit t$.}]
	\vspace{1em}	 
	\begin{tasks}
		\task Déterminer les réels $a$, $b$ et $c$ tels que $\displaystyle f(t)=\frac{a}{(t+3)^2} + \frac{b}{(t+3)^3} + \frac{c}{(t+3)^4}$.
		\sol{$a=1$, $b=-6$ et $c=8$}
		\task En déduire la valeur de $I$. 
		 \sol{Chaque terme est de la forme $\displaystyle \frac{u'}{u^n}$ avec $u=t+3$ et a pour primitive $\displaystyle \frac{1}{(n-1)u^{n-1}}$. Finalement, \hbox{$\displaystyle I=\frac{1}{4} + \frac{-9}{16} + \frac{7}{24} = -\frac{1}{48}$}.}
	\end{tasks}
\end{question}

%%%%%%%%%%%%%%%%%
%
% Intégrales doubles
%
%%%%%%%%%%%%%%%%%

\indent \exo{Intégrales doubles}


\begin{question}[name=~,subtitle={\normalfont Calculer les intégrales suivantes}]
	\begin{tasks}
		\task $\displaystyle I=\iint_D xy~ \ddroit x \ddroit y$ avec $D= \{ (x,y) \in \mathbb{R}^2~ |~ 0\leq x \leq 1 ~,~ 0 \leq y \leq 1-x \}$
		\sol{$\displaystyle I=\int_0^1 \left( \int_0^{1-x}xy~\ddroit y \right)\ddroit x = \int_0^1 \frac{1}{2}x(1-x)^2~\ddroit x  = \frac{1}{24}$
		}
		\task $\displaystyle  I=\iint_D \sin(x+y)~\ddroit x \ddroit y$ avec $D=\{ (x,y)\in\mathbb{R}^2 ~|~ x,y\geq0 \text{ et } x+y \leq \pi\}$
		\sol{$\displaystyle I=\int_{x=0}^{\pi} \int_{y=0}^{\pi-x}\sin(x+y) ~\ddroit y\ddroit x  = \int_{0}^{\pi}cos(x)+1 \ddroit x  = \pi$}
	\end{tasks}
\end{question}

\indent \exo{Calcul d'aires et de volumes}
\begin{question}[name=~,subtitle={\normalfont Utiliser le calcul intégral pour démontrer que la surface d'un disque de rayon $R$ vaut $\pi R^2$.}]
\textit{Indication : On pourra utiliser un système de coordonnées cylindriques et une surface élémentaire de dimension $\ddroit r\times r\ddroit\theta$.}
\end{question}
\sol{En sommant les surfaces élémentaires de $r=0$ à $r=R$ et de $\theta=0$ à $\theta=2\pi$, on couvre la surface d'un disque et : 
$$ \int_0^R \int_0^{2\pi} r\ddroit r\ddroit \theta = \pi R^2$$
}

\begin{question}[name=~,subtitle={\normalfont Démontrer que le volume d'un cylindre de rayon $R$ et de hauteur $H$ vaut $H \pi R^2$.}]
\end{question}
\sol{Même raisonnement : 
$$ \int_0^H \int_0^R \int_0^{2\pi} r\ddroit r\ddroit \theta \ddroit z= \pi R^2H$$}

\begin{question}[name=~,subtitle={\normalfont Démontrer que le volume d'une boule de rayon $R$ vaut $4 \pi R^3 /3$.}]
\textit{Indication : On pourra utiliser un système de coordonnées sphériques et un volume élémentaire de dimension $\ddroit r\times r\ddroit\theta \times r \sin(\phi)\ddroit \phi$. \\
\indent Autre méthode : Intégrer une ``pile'' de disques.}
\end{question}
\sol{Première méthode : 
$$V=\int_{r=0}^R \int_{\theta=0}^{2\pi} \int_{\phi=0}^{\pi} r^2 \sin\phi \ddroit r \ddroit \theta \ddroit \phi= \frac{4\pi R^3}{3}$$
Deuxième méthode :
$$V = \int_{-R}^R \pi(R^2-h^2) \ddroit h = \frac{4\pi R^3}{3}$$
}


%Application : Calcul de barycentre

%%%%%%%%%%%%%%%%%
%
% Intégrales curvilignes
%
%%%%%%%%%%%%%%%%%
%\indent \exo{Intégrales curvilignes}

%

%%%%%%%%%%%%%%%%%%%%%%%%%%%%%%%%%%%%%%%%%%%%%%%%%%%%%%%%%%%%%%%%%%%%%
%\begin{center}
%	\fbox{\Large \textsc{ Exercices supplémentaires}}
%\end{center}


%%%%%%%%%%%%%%%%%%%%%%%%%%%%%%%%%%%%%%%%%%%%%%%%%%%%%%%%%%%%%%
\end{document}