\section{Approche bayésienne}

Rappel des estimation de sources et du paramètre de régularisation proposés par Antoni 2012.\\

la probabilité est la traduction d'un état de connaissance du système.

\subsection{Formulation probabiliste du problème direct}

Le problème direct revient à exprimer la pression $\bm{p}$ au niveau de l'antenne de microphones en fonction du champ source $\bm{q}$ et du modèle de propagation acoustique $\bm{G}$.\\
Le champ source $q(\bm{r})$ peut est décomposé sur une base de $K$ fonctions spatiales $\phi_k(\bm{r})$ normalisées :

\begin{equation}
q(\bm{r}) = \bm{c} \cdot \bm{\phi}
\end{equation}
Les inconnues du problèmes sont donc les fonctions $\phi_k$, les coefficients $c_k$ qui dépendent des mesures et leur nombre $K$.\\

L'approche bayésienne propose de voir ces coefficients comme des variables aléatoires et d'étudier leur probabilité conditionnée aux mesures $[q(\bm{c},\bm{\phi})|\bm{p}]$. Si cette probabilité est élevée, ça signifie que les mesures expliquent précisément le champ source $q$. L'objectif est donc d'estimer ces variables de façon à ce qu'elles expliquent au mieux les mesures $\bm{p}$. Ces estimations de $\bm{\phi}$ et de $\bm{c}$ sont notées respectivement $\bm{\hat{\phi}}$ et $\bm{\hat{c}}$ telles que : 
\begin{equation}
	(\bm{\hat{c}},\bm{\hat{\phi}}) = \text{Argmax}[q(\bm{c},\bm{\phi})|\bm{p}]
\end{equation}

La loi de Bayes permet d'exprimer $[q(\bm{c},\bm{\phi})|\bm{p}]$ ainsi : 
\begin{equation}
[q(\bm{c},\bm{\phi})|\bm{p}]=\frac{[\bm{p}|q(\bm{c},\bm{\phi})][q(\bm{c},\bm{\phi})]}{[\bm{p}]}.
\end{equation}

En prenant le logarithme négatif de la quantité à maximiser, on peut définir une fonction coût à minimiser : 
\begin{equation}
	J(\bm{c},\bm{\phi}) = - \ln[q|\bm{p}] = -\ln[\bm{p}|q]-\ln[q]+\ln[\bm{p}]
\end{equation}

Le bruit de mesure $\bm{n}$  a une distribution gaussienne et par conséquent (?), $[\bm{p}|q]$ suit également une loi gaussienne. En introduisant la matrice de covariance $\mathbb{E}\{\bm{n}\bm{n}^*\}=\beta^{2}\bm{\Omega}_B$ ($\beta^2$ étant l'énergie moyenne du bruit, $\bm{\Omega}_B$ matrice connue selon la nature du bruit) : 
\begin{equation}
[\bm{p}|q] =\mathcal{N_c}(\bm{G}\bm{q},\beta^2\bm{\Omega}_B = \frac{   1   }{ \pi^M \beta^{2M} |\bm{\Omega}_N| } \exp \left( -\frac{1}{\beta^2} ||   \bm{p}  -  \bm{Gq}  ||^2_{\bm{\Omega}_N} \right)
\end{equation}

Une distribution gaussienne est également choisie pour la densité de probabilité des sources, de moyenne nulle et de variance $\mathbb{E}\{\bm{q}\bm{q}'\}=\alpha^2 \bm{\Omega}_q$ : 
\begin{equation}
	[\bm{q}] = \mathcal{N_c}(\bm{0}, \alpha^2\bm{\Omega}_q) = \frac{  1  }{  \pi^K\alpha^{2K} |\bm{\omega}_q|} \exp \left( -\frac{1}{\alpha^2}   ||\bm{q}||^2_{\bm{\Omega}_q} \right)
\end{equation}

vraisemblance : adéquation entre une distribution observée (sur échantillon) et la loi de proba qui décrit la population dont est issu l'échantillon 
fonction de vraisemblance : la vraisemblance varie  en fonction des paramètres de la loi choisie. Paramètre s'appelle généralement $\theta$. Sert donc a ajuster des observations à une loi.




\subsection{Régularisation bayésienne}

Le paramètre de régularisation peut être estimé par une approche bayésienne. 
objectif : développer un formalisme généralisant les diverses approches développées pour chaque contexte d'imagerie ; prendre en compte le maximum d'informations a priori connues.

Les informations connues a priori peuvent être incertaines et décrites par une densité de probabilité. L'approche bayésienne permet donc de prendre en compte ces informations "incertaines".\\
Dans cette approche, les paramètres inconnus (sources acoustiques) sont exprimés en fonction des données de mesures selon la loi de Bayes : 
\begin{equation}
[x|y] = \frac{[y|x][x]}{[y]}
\end{equation}
où la notation $[y]$ est la densité de probabilité de la variable $x$ et $[x|y]$ est la densité de probabilité de $x$ conditionnée à la variable $y$





Calcul de l'intervalle de confiance ??
	
