\documentclass[portrait,final,a0paper]{baposter}

\usepackage[francais]{babel}
\usepackage[utf8]{inputenc}
\usepackage[T1]{fontenc}

\usepackage{calc}
\usepackage{graphicx}
\usepackage{amsmath}
\usepackage{amssymb}
\usepackage{relsize}
\usepackage{multirow}
\usepackage{rotating}
\usepackage{bm}
\usepackage{enumitem}
\usepackage{url}
\usepackage{booktabs}
%\usepackage{multicol}
\usepackage{xcolor}
\usetikzlibrary{calc}

\usepackage{tabularx}


%\usepackage{times}
\usepackage{helvet}
%\usepackage{bookman}
%\usepackage{palatino}
\renewcommand*{\familydefault}{\sfdefault}

\newcommand{\captionfont}{\footnotesize}

\graphicspath{{images/}}

\setlength{\columnsep}{1.5em}
\setlength{\columnseprule}{0mm}

\newcommand{\marktikz}[1]{ \tikz[remember picture,overlay]\node (#1) {};}

%%%%%%%%%%%%%%%%%%%%%%%%%%%%%%%%%%%%%%%%%%%%%%%%%%%%%%%%%%%%%%%%%%%%%%%%%%%%%%%%
% Save space in lists. Use this after the opening of the list
%%%%%%%%%%%%%%%%%%%%%%%%%%%%%%%%%%%%%%%%%%%%%%%%%%%%%%%%%%%%%%%%%%%%%%%%%%%%%%%%
\newcommand{\compresslist}{%
\setlength{\itemsep}{1pt}%
\setlength{\parskip}{0pt}%
\setlength{\parsep}{0pt}%
}



\newcommand*\circled[1]{\tikz[baseline=(char.base)]{
            \node[shape=circle,draw,inner sep=2pt,color=main,fill=main!10, line width=1pt] (char) {#1};}}

%%%%%%%%%%%%%%%%%%%%%%%%%%%%%%%%%%%%%%%%%%%%%%%%%%%%%%%%%%%%%%%%%%%%%%%%%%%%%%
%%% Begin of Document
%%%%%%%%%%%%%%%%%%%%%%%%%%%%%%%%%%%%%%%%%%%%%%%%%%%%%%%%%%%%%%%%%%%%%%%%%%%%%%

\begin{document}

%%%%%%%%%%%%%%%%%%%%%%%%%%%%%%%%%%%%%%%%%%%%%%%%%%%%%%%%%%%%%%%%%%%%%%%%%%%%%%
%%% Here starts the poster
%%%---------------------------------------------------------------------------
%%% Format it to your taste with the options
%%%%%%%%%%%%%%%%%%%%%%%%%%%%%%%%%%%%%%%%%%%%%%%%%%%%%%%%%%%%%%%%%%%%%%%%%%%%%%
% Define some colors


\definecolor{green}{rgb}{0.32,0.67,0.39}
%\definecolor{main}{rgb}{0 0.627 0.588} %canard
\definecolor{main}{rgb}{0 0.439  0.753} %bleu

%\hyphenation{resolution occlusions}
%%
\begin{poster}%
  % Poster Options
  {
  % Show grid to help with alignment
  grid=false,
  % Column
  columns=4,
  colspacing=1em,
  % Color style
  background=plain,%%shadetb,
  bgColorOne=gray!12,%lightestgreen!80,
  bgColorTwo=white,
  borderColor=white,%darkgreen,
  headerColorOne=white,%darkgreen,
  %headerColorTwo=lightgreen,
  headerFontColor=main,%white,
  boxColorOne=white,%lightestgreen,
  %boxColorTwo=lightgreen,
  % Format of textbox
  textborder=roundedleft,%faded,
  % Format of text header
  eyecatcher=true,
  headerborder=closed,
  headerheight=0.15\textheight,
  %textfont=\sc,% An example of changing the text font
  headershape=roundedright,
  headershade=plain,%shadelr,
  headerfont=\Large\bfseries,%\textsc, %Sans Serif
  textfont={\setlength{\parindent}{1.5em}},
  boxshade=plain,
  linewidth=2pt
  }
  % Eye Catcher
  {
  \begin{minipage}{3cm}
  	\centering
 	 \includegraphics[width=3cm]{logo/LABEX_CELYA.jpg} \\
 	  \includegraphics[trim={0 3cm 0 3cm},clip=true,width=3cm]{logo/logo_ADAPT.png}\\~\\

\end{minipage}
  }
  % Title
 {
%\begin{tikzpicture}[overlay, remember picture, inner sep=0pt, outer sep=0pt]
%  %\fill [white] ([yshift=-3cm]current page.north west) rectangle (current page.north east);
%\end{tikzpicture}
%\begin{minipage}{\linewidth}
%	\vspace{0.5cm}
%	\includegraphics[height=1cm]{logo/LABEX_CELYA.jpg} \hfill \includegraphics[height=1cm]{logo/logo_lmfa.pdf} \hfill \includegraphics[height=1cm]{logo/LVA_compact_couleur_transparent.png} \hfill   \includegraphics[trim={0 3cm 0 3cm},clip=true,height=1cm]{logo/logo_ADAPT.png}\\
%\end{minipage} 
 \textcolor{main}{\textbf{ Débruitage de la matrice interspectrale \\ pour l'étude des sources aéroacoustiques}}}
  % Authors
  { ~\\A. Dinsenmeyer\textsuperscript{1,2}, Q. Leclère\textsuperscript{1}, J. Antoni\textsuperscript{1}, E. Julliard\textsuperscript{3}}
  % University logo
  {
\begin{minipage}{3cm}
\centering
 \includegraphics[width=2.5cm]{logo/LVA_compact_couleur.jpg}\\
 \includegraphics[width=2.5cm]{logo/logo_lmfa.pdf} \\~\\
\end{minipage}
  }
  
 
%%%%%%%%%%%%%%%%%%%%%%%%%%%%%%%%%%%%%%%%%%%%%%%%%%%%%%%%%%%%%%%%%%%%%%%%%%%%%%
%%% Now define the boxes that make up the poster
%%%---------------------------------------------------------------------------
%%% Each box has a name and can be placed absolutely or relatively.
%%% The only inconvenience is that you can only specify a relative position 
%%% towards an already declared box. So if you have a box attached to the 
%%% bottom, one to the top and a third one which should be in between, you 
%%% have to specify the top and bottom boxes before you specify the middle 
%%% box.
%%%%%%%%%%%%%%%%%%%%%%%%%%%%%%%%%%%%%%%%%%%%%%%%%%%%%%%%%%%%%%%%%%%%%%%%%%%%%%



\headerbox{Contexte}{name=contexte}{
Verrous scientifiques et technologiques
 }
 
 \headerbox{Méthode proposée}{name=methode,below=contexte,column=0,span=4}{
 
 \begin{minipage}[t][10cm][t]{0.3\textwidth}
	\centering
	\resizebox{0.8cm}{!}{\circled{\textbf{1}}} \marktikz{step1out}  \\[1ex]
	 {\bfseries Choisir un modèle statistique}\\
	 $\displaystyle \bm{M\left( \bm{\theta} \right)}$
	 \vspace{1em}

	 
	 
	 \fbox{
	\begin{tabular}{m{1ex} m{1ex} m{3cm} m{1ex}  m{2cm}}
	$\bm{p}$ & $=$ & \centering$\bm{L \alpha c}$& $+$ &  \parbox{\linewidth}{ \centering $\bm{n}$} \\[1ex]
	 & $=$  & \tikz{
	 	\draw[rectangle,main,line width=1pt,fill=main!10] rectangle ++(0.5cm,-2cm); 
		\draw[rectangle,main,line width=1pt] ++(0.6cm,-0.75cm) rectangle ++(0.5cm,-0.5cm);
		\draw[main,line width=1pt] ++(0.6cm,-0.75cm) to  ++(0.5cm,-0.5cm);
		\draw[main,rectangle,line width=1pt,fill=main!10] ++(1.2cm,-0.75cm) rectangle ++(2cm,-0.5cm);
		}
		 & $+$ & 
		 \tikz{
		\draw[rectangle,main,line width=1pt] rectangle ++(2cm,-2cm); 
		\draw[line width=1pt,main] (0,0)to ++(2cm,-2cm);
		}\\
		 & &  \parbox{\linewidth}{ $\underbrace{\hspace{\linewidth}}$\\ \centering \scriptsize Matrice parcimonieuse} & & \parbox{\linewidth}{$\underbrace{\hspace{\linewidth}}$\\ \centering \scriptsize Bruit décorrélé}
	\end{tabular}
	}


	\end{minipage}
	\hfill
	 \begin{minipage}[t][10cm][t]{0.3\textwidth}
	\centering
	\marktikz{step2in} 
	\resizebox{0.8cm}{!}{\circled{\textbf{2}}}
	\marktikz{step2out} \\[1ex]
	{ \bfseries Choisir des distributions à priori}
	\vspace{1em}
	
	\fbox{
	
	
	}
	
	\end{minipage}
	\hfill
	 \begin{minipage}[t][10cm][t]{0.2\textwidth}
		\centering
		\marktikz{step3in}\resizebox{0.8cm}{!}{\circled{\textbf{3}}}  \marktikz{step3out}  \\[1ex]
		Faire tourner un échantillonneur de Gibbs
	\end{minipage}\\
	\hfill
	\flushright \begin{minipage}[t][10cm][t]{0.5\textwidth}
		\centering
		\marktikz{step4in}   \resizebox{0.8cm}{!}{\circled{\textbf{4}}} \marktikz{step4out} \\[1ex]
		Reconstruire la CSM débruitée
	\end{minipage}

	\tikz[remember picture,overlay]{
		\draw[->,line width=1pt, main,>=latex] ($ (step1out.east)+(1ex,1ex)$) to [out=30,in=150] ( $ (step2in.west) + (-1.5ex,1ex) $ );%
		\draw[->,line width=1pt, main,>=latex] ($ (step2out.east)+(1ex,1ex)$) to [out=30,in=150] ( $ (step3in.west) + (-1.5ex,1ex) $ );%
		\draw[->,line width=1pt, main,>=latex] ($ (step3out.east)+(1ex,1ex)$) to [out=-30,in=60] ( $ (step4out.east) + (1.5ex,1ex) $ );%
	
	}

 }
 
 
 \headerbox{Application à l'imagerie}{name=imagerie,column=0,span=4,below=methode}{
 
 }
 
 \headerbox{Analyse}{name=analyse,column=0, span=2,below=imagerie}{
 
 }
 
 \headerbox{Perspectives}{name=perspective,column=2, span=2,below=imagerie}{
 
 }
 
 







%%%%%%%%%%%%%%%%%%%%%%%%%%%%%%%%%%%%%%%%%%%%%%%%%%%%%%%%%%%%%%%%%%%%%%%%%%%%%%
  \headerbox{}{name=about,column=0,span=4,below=perspective}{
%%%%%%%%%%%%%%%%%%%%%%%%%%%%%%%%%%%%%%%%%%%%%%%%%%%%%%%%%%%%%%%%%%%%%%%%%%%%%%
  \noindent
  \small
  Contact : \url{alice.dinsenmeyer@insa-lyon.fr}\\
  \textsuperscript{1}Laboratoire Vibrations Acoustique, Villeurbanne ; 
\textsuperscript{2}Laboratoire de Mécanique des Fluides et Acoustique, Écully ; 
\textsuperscript{3}Airbus, Toulouse
%  \smaller
%  
%  
%  \begin{minipage}{\linewidth}
%      \begin{minipage}{0.45\linewidth}
%          Poster by Lucas Bourneuf, in \LaTeX\\
%          Source available on github
%      \end{minipage}\hfill%
%      \begin{minipage}{0.375\linewidth}
%          \indent{}POGG is also on the web:\\
%          \url{http://pogg.genouest.org}
%      \end{minipage}\hfill%
%      \begin{minipage}{0.125\linewidth}
%
%      \end{minipage}
%  \end{minipage}
%  \begin{center}
%      \url{https://github.com/Aluriak/POGG_poster}
%  \end{center}
  }



\end{poster}

\end{document}
