\documentclass[12pt]{article}
\setlength{\columnsep}{2cm}

%\usepackage{cite} 
\usepackage[round,authoryear,numbers]{natbib}
\usepackage[french]{babel}
\usepackage[utf8]{inputenc}
\usepackage{graphicx} %pour mettre des figures dans multicol avec l'environnement figure*

\usepackage[T1]{fontenc} % Use 8-bit encoding that has 256 glyphs
%\linespread{1.05} % Line spacing - Palatino needs more space between lines
%\usepackage{microtype} % Slightly tweak font spacing for aesthetics

\usepackage[hmarginratio=1:1,top=10mm, right=12mm]{geometry} % Document margins
\usepackage[textfont=it]{caption} % Custom captions under/above floats in tables or figures
\usepackage{booktabs} % Horizontal rules in tables
%\usepackage{float} % Required for tables and figures in the multi-column environment - they need to be placed in specific locations with the [H] (e.g. \begin{table}[H])
\usepackage{hyperref} % For hyperlinks in the PDF

\usepackage{bm}
\usepackage{amsfonts}
\usepackage{amsmath}
\usepackage{amssymb}
\usepackage{tabularx}

\usepackage{titlesec} % Allows customization of titles
\renewcommand\thesection{\Roman{section}} % Roman numerals for the sections
\renewcommand\thesubsection{\arabic{section}.\arabic{subsection}} % Roman numerals for subsections
\titleformat{\section}[block]{\bfseries\centering}{\thesection.}{1em}{}[{\titlerule[1.2pt]}] % Change the look of the section titles
\titleformat{\subsection}[block]{\bfseries}{\thesubsection.}{1em}{} % Change the look of the section titles
\renewcommand\thesubsubsection{\small{\arabic{section}.\arabic{subsection}.\arabic{subsubsection}}}
\titleformat{\subsubsection}[block]{\bfseries}{\thesubsubsection}{0.5em}{}
\titleformat*{\paragraph}{\vspace{-0.3cm}\small\bfseries}

\newcommand{\tbullet}{$\vcenter{\hbox{\tiny$\bullet$}}~$}

\usepackage{fancyhdr} % Headers and footers
\pagestyle{fancy} % All pages have headers and footers
\fancyhead{} % Blank out the default header
\fancyfoot{} % Blank out the default footer
\renewcommand{\headrulewidth}{0pt} %pour enlever la ligne du header
%\fancyhead[C]{titre, date, noms...	} % Custom header text
%\fancyfoot[RO,RE]{\thepage} % Custom footer text
%\fancyfoot[LO,LE]{A. DINSENMEYER, \today}
%\renewcommand{\footrulewidth}{0.4pt} 

%modif des espacement avant et après l'environnement equation
\let\oldequation=\equation
\let\endoldequation=\endequation
\renewenvironment{equation}{\vspace{-0.2cm}\begin{oldequation}}{\vspace{-0.2cm}\end{oldequation}}
 
%agrandissement de la zone de texte
%\addtolength{\oddsidemargin}{-1cm}
%\addtolength{\evensidemargin}{-1cm}
%\addtolength{\textwidth}{2cm}
%\addtolength{\topmargin}{-0.7cm}
\addtolength{\textheight}{1cm}

\usepackage{color}
\usepackage[color=blue!20]{todonotes}
\usepackage{mathtools}

%pour écrire du pseudo code :
\usepackage{algorithm}
\usepackage{algorithmic}

\usepackage{hyperref}
\hypersetup{
     colorlinks   = true,
     citecolor    = blue!90
}

\newcommand{\dd}{\partial}
\newcommand{\ok}{ \textcolor{orange}{\bfseries \textsc ok }}


\usepackage{subcaption}
\usepackage{tabulary}
\usepackage{pgfplots} 


%----------------------------------------------------------------------------------------
%	TITLE SECTION
%----------------------------------------------------------------------------------------

\title{\vspace{-12mm}\fontsize{14pt}{14pt}\selectfont\textbf{Préparation des JJCAB 2018 - Notes}} % Article title

%\author{
%\large{Alice \textsc{Dinsenmeyer}}\\[2mm] % Your name %\thanks{}
%\normalsize University of California \\ % Your institution
%\normalsize \href{mailto:john@smith.com}{john@smith.com} % Your email address
%\vspace{-5mm}
%}
\date{\vspace{-1cm}
}

%----------------------------------------------------------------------------------------

\begin{document}
\maketitle 
\section*{Contenu de la présentation flash}

\paragraph{Contexte : }~\\
\begin{itemize}
        \item CSM (antenne, stationnaire)
        \item Bruit gaussien, aéro, bruit TBL $>$ acoustique
        \item Suppression de la diagonale
        \item Problème inverse, imagerie
\end{itemize}
\paragraph{Raisonnements scientifiques : }~\\
\begin{itemize}
        \item problème d'optimisation
        \item paramètres du modèle : variables aléatoires dont on cherche la densité de probabilité
        \item maximisation de la distribution a posteriori
\end{itemize}

\section*{Contenu du poster}
\begin{itemize}
        \item Contexte
        \item Méthode PFA
        \item Cas numérique : imagerie  (beamforming ou  autre ?)
        \item Application industrielle (Airbus est ok, si pas de niveau absolu, ni fréquence absolu. Parler d'avion de ligne et faire un schéma vague)
\end{itemize}


\section*{Résumé}

Avec l'apparition des MEMS et la diminution globale du coût des capteurs, les acquisitions multivoies se généralisent, notamment dans le domaine de l'identification de sources acoustiques. La qualité de la localisation et de la quantification des sources peut être dégradée par la présence de bruit de mesure ambiant ou induit par le système d'acquisition. En particulier, dans le cas de mesures en présence d'un écoulement, le bruit de couche limite turbulente peut être supérieur au niveau des sources à caractériser, et il devient nécessaire de traiter les acquisitions pour extraire la contribution des sources de celle du bruit.

 Pour cela, on propose de décomposer la matrice spectrale mesurée en la somme d'une matrice signal et d'une matrice bruit. Cette décomposition exploite les propriétés statistiques de ces deux matrices. Comme le signal sources est corrélé sur les capteurs, le rang de la matrice spectrale associée se limite au nombre sources décorrélées équivalentes. De plus, le bruit aléatoire est supposé avoir une longueur de corrélation plus faible que celle du signal des sources, ce qui peut être approché par une matrice spectrale diagonale. En s'appuyant sur ce modèle, le débruitage est traité comme un problème d'optimisation bayésienne. 
 
À l'aide d'un cas numérique, les performances, en terme d'imagerie acoustique, sont comparées à la méthode classique de débruitage qui consiste à supprimer les éléments diagonaux de la matrice spectrale bruitée.
Dans un second temps, la méthode est appliquée à des données industrielles particulièrement bruitées, provenant de mesures microphoniques effectuées sur le fuselage d'un avion de ligne en vol.










\end{document}
